\documentclass[12pt]{exam}
\usepackage[spanish]{babel}
\usepackage{amsmath} % Para notación matemática

\usepackage{amssymb}
\usepackage{multicol} % Para crear columnas
\usepackage[left=10mm, right= 20mm, bottom= 20mm, top= 35mm, headsep=10mm, headheight=0mm]{geometry}
\usepackage{setspace}  
\usepackage{graphicx}
\renewcommand\partlabel{\thepartno)}
%%% DEFINICIONES PARA MODIFICAR APARIENCIA DEL ENCABEZADO%%%
\usepackage{etoolbox}
\makeatletter
\patchcmd{\@fullhead}{\hrule}{\hrule\vskip2pt\hrule height 2pt}{}{}
\patchcmd{\run@fullhead}{\hrule}{\hrule\vskip2pt\hrule height 2pt}{}{}
\makeatother
\pagestyle{headandfoot}
%\runningheadrule
\firstpageheadrule
\firstpageheader {\escuela \\ Curso: \curso}
                 {  \tpnombre \\ Matemática }
                  {\lugar \\  \today       }       
%\runningheader{\includegraphics[width=0.1\textwidth]{logo.png}}
%{Matemática:Trabajo Práctico}
%{\today}
\firstpagefooter{Prof: \textit{Lucas H.Trejo}}{}{\thepage}
\runningfooter{Prof: \textit{Lucas H.Trejo}}{}{Pag. \thepage\ of \numpages}

%%%%%%%%%%%%%%%%%%%%%%%%%%%%%%%%%%%%%%%%%%%%%%%%%%%%%%%%%%%%%%%%%%%%%%%%%%%%%%%
                                       %%% REEMPLAZAR LOS PARÁMETROS AQUÍ    %%  
\newcommand{\escuela}{EES Nº 4}                                             %%             
\newcommand{\curso}{  3º año}                                                  %%             
\newcommand{\tpnombre}{ PORCENTAJES }                     %%
\newcommand{\lugar}{City Bell}                                               %%
%%%%%%%%%%%%%%%%%%%%%%%%%%%%%%%%%%%%%%%%%%%%%%%%%%%%%%%%%%%%%%%%%%%%%%%%%%%%%%%



\begin{document}

\begin{questions}
    \question Calcular los porcentajes indicados:
     
    \begin{parts}
      \begin{multicols}{3}
        
      
        \part El 35\% de 4 800 es .....
        \part El 75\% de 4 800 es .....
        \part El 12\% de 7 000 es .....
        \part El 50\% de 30 000 es .....
        \part El 25\% de 1 000 es .....
        \part El 25\% de 10 000 es .....
        \part El 30\% de 90 es .....
        \part El 30\% de 1 000 es .....
        \part El 77\% de 7 200 es .....
        \part El 3\% de 16 000 es .....
        \part El 21\% de 300 000 es .....
        \part El 18\% de 40 000 es .....
        \part El 95\% de 2 400 es .....
        \part El 12\% de 700 es .....
    
        \part El 2,5\% de 20 000 es .....
        \part El 1,5\% de 6 000 es .....
        \part El 0,5\% de  2 000 es .....
        \part El 0,14\% de 1 000 000 es .....
        \part El 0,01\% de 1 000 000 es .....
        \part El 0,02\% de 5 000 es .....
    \end{multicols}    
    \end{parts}
        
    \question ¿A qué número corresponden los siguientes porcentajes? 
   
    \begin{parts}
        \begin{multicols}{3}
            
        
    
        \part 5 000 es el 35\% de .....
        \part 4 000 es el 30\% de .....
        \part 6 000 es el 35\% de .....
        \part 2 500 es el 10\% de .....
        \part 5 000 es el 30\% de .....
        \part 1 000 es el 20\% de .....
        \part 2 000 es el 21\% de .....
        \part 10 000 es el 21\% de .....
        \part 1 500 es el 1,5\% de .....
        \part 800  es el 8\% de .....
        \part 1352 es el 26\% de .....
        \part 2 700 es el 9\% de .....
        \part 600 es el 12\% de .....
        \part 120 es el 3\% de .....
    \end{multicols}
    \end{parts}

    \question Completar con el número que corresponda en cada caso:
    \begin{parts}
        \begin{multicols}{3}
            
        
    
        \part 460 es el ....\% de 9 200
        \part 516 es el ....\% de 600
        \part 9021 es el ....\% de 9 700
        \part 7047 es el ....\% de 8 100
        \part 3618 es el ....\% de 5 400
        \part 2160 es el ....\% de 3 600
        \part 17 es el ....\% de 100
        \part 1500 es el ....\% de 6 000
        \part 713 es el ....\% de 3 100
        \part 1266 es el ....\% de 7 000
        \part 1638 es el ....\% de 7 800
        \part 1080 es el ....\% de 6 000
        \part 0 es el ....\% de 4 500
        \part 2400 es el ....\% de 8 000
        \part 1298 es el ....\% de 2 200
        \part 396 es el ....\% de 1 800
        \part 3036 es el ....\% de 6 900
        \part 792 es el ....\% de 2 400
        \part 5568 es el ....\% de 5 800
        \part 3050 es el ....\% de 5 000
    \end{multicols}
    \end{parts}
    
    \textbf{Empecemos con los problemas:}
        
        
        \question Si en una compra de \$12 000, me hacen un 5\% de descuento ¿Cuál es el monto final de la compra?
        \question Si en una compra de \$1 600, me hacen un 30\% de descuento por pago con cuenta DNI ¿Cuál es el monto final de la compra?
        
        \question Si gasto 8 500\$ en una verdulería, y tiene 35\% de descuento con una aplicación ¿Cuánto
                  me reintegra la app ?  
        
 
        \question El i.v.a. es un impuesto que equivale al 21\% del valor de la compra. ¿Cuánto paga de i.v.a. 
                  alguien que gasta \$ 200 000 ?  
        
        \question Si en una compra de \$ 15 000 me devuelven \$ 3 000 ¿Cuál fue el porcentaje de descuento? 
        
        \question Si quiero llenar un frasco con 300 ml de alcohol al 70 \% ( 70\% alcohol y 30\% agua)
                  ¿Cuánto alcohol y cuánta agua debo colocar en el frasco? 
        
        \question Si un kilo de arroz costaba \$ 1 200 el año pasado, y ahora cuesta \$ 2 400 ¿Qué porcentaje aumentó?
        
        \question El día 7 de marzo, el precio del kilo de harina era de \$900. El día 7 de mayo, el precio del kilo de harina ya era de \$1 200. 
                  ¿Qué porcentaje aumentó el kilo de harina en esos 2 meses?
        \question Si un pasaje pasó de costar \$ 270 a \$371 ¿ Qué porcentaje aumentó?
        
        \question Para una masa de pan se calcula que la cantidad de sal sea el 2\% del peso de la harina. 
        \begin{parts}
           \part ¿ Cuánta sal le pondría a una masa con 500 gramos de harina?     
           \part ¿ Y para 2 kilos ?     
        
        \end{parts}     
        
        \question Un alimento para mascota tiene el 26\% de proteina. 
                 Si servimos una porión de 150 gramos ¿cuántos gramos de proteina tendrá?
    

\end{questions}

\end{document}