\documentclass[12pt]{exam}
\usepackage[spanish]{babel}
\usepackage{amsmath} % Para notación matemática

\usepackage{amssymb}
\usepackage{multicol} % Para crear columnas
\usepackage[left=10mm, right= 20mm, bottom= 20mm, top= 35mm, headsep=10mm, headheight=0mm]{geometry}
\usepackage{setspace}  
\usepackage{graphicx}
\renewcommand\partlabel{\thepartno)}
%%% DEFINICIONES PARA MODIFICAR APARIENCIA DEL ENCABEZADO%%%
\usepackage{etoolbox}
\makeatletter
\patchcmd{\@fullhead}{\hrule}{\hrule\vskip2pt\hrule height 2pt}{}{}
\patchcmd{\run@fullhead}{\hrule}{\hrule\vskip2pt\hrule height 2pt}{}{}
\makeatother
\pagestyle{headandfoot}
%\runningheadrule
\firstpageheadrule
\firstpageheader {\escuela \\ Curso: \curso}
                 {  \tpnombre \\ Matemática }
                  {\lugar \\  \today       }       
%\runningheader{\includegraphics[width=0.1\textwidth]{logo.png}}
%{Matemática:Trabajo Práctico}
%{\today}
\firstpagefooter{Prof: \textit{Lucas H.Trejo}}{}{\thepage}
\runningfooter{Prof: \textit{Lucas H.Trejo}}{}{Pag. \thepage\ of \numpages}

%%%%%%%%%%%%%%%%%%%%%%%%%%%%%%%%%%%%%%%%%%%%%%%%%%%%%%%%%%%%%%%%%%%%%%%%%%%%%%%
                                       %%% REEMPLAZAR LOS PARÁMETROS AQUÍ    %%  
\newcommand{\escuela}{EES Nº 4}                                             %%             
\newcommand{\curso}{  3º año}                                                  %%             
\newcommand{\tpnombre}{ PORCENTAJES }                     %%
\newcommand{\lugar}{City Bell}                                               %%
%%%%%%%%%%%%%%%%%%%%%%%%%%%%%%%%%%%%%%%%%%%%%%%%%%%%%%%%%%%%%%%%%%%%%%%%%%%%%%%



\begin{document}

\begin{questions}
    \question Completar los espacios en blanco:
     
    \begin{parts}
      \begin{multicols}{3}
        
      
        \part El 58\% de 4800 es .....
        \part El 73\% de 4400 es .....
        \part El 12\% de 1700 es .....
        \part El 81\% de 2315 es .....
        \part El 78\% de 5200 es .....
        \part El 75\% de 2200 es .....
        \part El 68\% de 210 es .....
        \part El 90\% de 4100 es .....
        \part El 77\% de 7200 es .....
        \part El 3\% de 4400 es .....
        \part El 48\% de 9512 es .....
        \part El 7\% de 9300 es .....
        \part El 93\% de 2300 es .....
        \part El 9\% de 215 es .....
    
        \part El 39\% de 7600 es .....
        \part El 86\% de 122 es .....
        \part El 58\% de 7800 es .....
        \part El 17\% de 140 es .....
        \part El 32\% de 4300 es .....
        \part El 1\% de 710 es .....
    \end{multicols}    
    \end{parts}
        
    \question Completar 
   
    \begin{parts}
        \begin{multicols}{3}
            
        
    
        \part 7505 es el 79\% de .....
        \part 850 es el 42\% de .....
        \part 3312 es el 46\% de .....
        \part 2426 es el 41\% de .....
        \part 4392 es el 72\% de .....
        \part 368 es el 92\% de .....
        \part 1829 es el 59\% de .....
        \part 1980 es el 55\% de .....
        \part 3337 es el 71\% de .....
        \part 682 es el 11\% de .....
        \part 1352 es el 26\% de .....
        \part 297 es el 9\% de .....
        \part 528 es el 12\% de .....
        \part 56 es el 7\% de .....
    \end{multicols}
    \end{parts}

    \question Completar:
    \begin{parts}
        \begin{multicols}{3}
            
        
    
        \part 460 es el ....\% de 5400
        \part 516 es el ....\% de 600
        \part 9021 es el ....\% de 9700
        \part 7047 es el ....\% de 8100
        \part 3618 es el ....\% de 5400
        \part 2160 es el ....\% de 3600
        \part 17 es el ....\% de 100
        \part 1995 es el ....\% de 5400
        \part 713 es el ....\% de 3100
        \part 1266 es el ....\% de 7000
        \part 5226 es el ....\% de 7800
        \part 1080 es el ....\% de 6000
        \part 0 es el ....\% de 4500
        \part 2400 es el ....\% de 8000
        \part 1298 es el ....\% de 2200
        \part 396 es el ....\% de 1800
        \part 3036 es el ....\% de 6900
        \part 792 es el ....\% de 2400
        \part 5568 es el ....\% de 5800
        \part 3050 es el ....\% de 5000
    \end{multicols}
    \end{parts}
    
    \textbf{Empecemos con los problemas:}

     
        \question Si en una compra de \$120, me hacen un 5\% de descuento ¿Cuál es el monto final de la compra?
        \question Si en una compra de \$160, me hacen un 10\% de descuento por pago en efectivo, y un 5\% de descuento por ser cliente del negocio. ¿Cuál es el monto final de la compra?
        
        \textit{Voy a una Disquería a comprar CDs y me dicen que el precio de cada uno es de \$22, a su vez, si compro 10 o más de 10 me hacen un 8\% de descuento, y si compro más de 15, me hacen un 12\% de descuento.}
        
        \question Si compro 11 CDs. ¿Cuánto me costarían?
        \question ¿Cuál sería el precio final unitario de cada CD si compro 20?
        \question ¿Qué descuento me tienen que hacer, como mínimo, para la compra de más de 10 CDs para que comprar 11 me salga menos que comprar 10 CDs?
        
        \question ¿Qué descuento me tendrían que hacer en una compra de \$160, para que el monto final sea \$120?
        \question Si tengo que comprar libros, cuyo precio de lista es de \$8. Tengo un total de \$490. ¿Qué descuento me tendrían que hacer para que me alcance para comprar 70 libros?
        
        \question El día 7 de marzo, el precio del kilo de harina era de \$0,90. El día 7 de mayo, el precio del kilo de harina ya era de \$1,20. ¿Qué porcentaje aumentó el kilo de harina en esos 2 meses?
        \question Si el último mes el kilo de arroz aumentó un 20\%, y ahora está a \$1,50 ¿Cuánto estaba el mes pasado?
        \question Si este mes el kilo de fideos aumentó un 10\%, y ahora está a \$1,20 ¿Cuánto estaba el mes pasado?
        \question Si dicen que en el último mes el kilo de azúcar aumentó un 10\% y el anterior había aumentado un 20\%. Ahora está \$0,66. ¿Cuánto estaba hace dos meses?
        \question Si dicen que en el último mes el kilo de yerba aumentó un 25\% y el anterior había aumentado un 15\%. Ahora está a \$0,60. ¿Cuánto estaba el mes pasado?
        
        \question En 8º A, hay 5 chicos de Boca, si en total hay 20 chicos y de ellos 18 son hinchas de algún club. ¿Qué porcentaje de los chicos de 8º A son hinchas de Boca?
        \question En 8º A, hay 6 chicos de River, si en total hay 20 chicos y de ellos 18 son hinchas de algún club. ¿Qué porcentaje de los chicos de 8º A, que son hinchas de algún club, son hinchas de River?
        \question Mariano juega al fútbol, todos los sábados, y un miércoles por medio. ¿Cuál es el porcentaje de días en el año que Mariano juega al fútbol?
     
    

\end{questions}

\end{document}